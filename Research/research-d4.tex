\documentclass{article}
\usepackage{amsmath}
\usepackage{cjhebrew}
\usepackage[LGR,T1]{fontenc}
\newcommand{\donaldiv}{\textgreek{Ντόναλντ Δ }}
\newcommand{\textgreek}[1]{\begingroup\fontencoding{LGR}\selectfont#1\endgroup}\newcounter{example}[section]
\newenvironment{example}[1][]{\refstepcounter{example}\par\medskip
   \noindent \textbf{Example~\theexample. #1} \rmfamily}{\medskip}

\title{Gauss-Bonet Theorem}
\author{\textgreek{Για Ντόναλντ Ιορδανιδου Δ}}
\begin{document}
\maketitle

The Gauss-Bonnet Theorem is a formula linking the curvature of a surface to its topology.
\begin{equation}
    \int_M K\,dA + \int_{\partial M} k_g \,ds = 2\pi\chi(M)
\end{equation}

\section{Descriptions}
There are a few functions within the theorem that would require descriptions unto themselves.
In this section, they will be explained.

\subsection{Euler's Characteristic}
It wouldn't be math if Euler wasn't hiding in there somewhere (\<'wyy>). 
Euler's characteristic, represented as a function with $\chi(M)$, is a function that tracks the number of objects of each dimension. 
A \donaldiv description of this would be, for $k_i$ being the number of $i-dimensional$ components of the overall object:
\begin{equation}
    \chi(M) = \overunderset{\infty}{i = 0}{\Sigma} k_i (-1)^i 
\end{equation}

\begin{example}\\
    $\chi(line) = 2 - 1 = 1$\\
    $\chi(5\ point\ star) = 10 - 10 + 2 = 2$\\
    $\chi(dodecahedron) = 20 - 30 + 12 = 2$\\
    $\chi(tesseract) = 16 - 32 + 18 - 2 = 0$
\end{example}

\section{Related Theorems}
\subsection{Theoreme de Descartes}
René n'était pas un philosopheur. 
Il était un mathématicien qui a appliqué les règles des preuves de mathématiques à des règles qu'il s'est inventé.
``Je pense donc je suis'' n'était pas une conclusion, c'était un postulat q'il s'est donné.
Qui autre d'un mathématicien pourrait nous donner le plan cartésien. 

% https://e.math.cornell.edu/people/belk/differentialgeometry/Outline\%20-\%20The\%20Gauss\%20Map.pdf
% https://en.wikipedia.org/wiki/Descartes\%27_theorem
% https://www2.math.upenn.edu/~shiydong/Math501X-7-GaussBonnet.pdf

\end{document}